\section{Command Cheat Sheet}
  \begin{center}
    \begin{tabular}{|p{7.5cm}|p{7.5cm}|}
    \hline
      \textbf{Command} & \textbf{Result} \\\hline
      \texttt{\pytool new problem.xml -i} & 
        Create a new problem file called \texttt{problem.xml}, and
        specify its topics and types using the \texttt{edit} tool \\\hline
      \texttt{\pytool edit problem.xml} &
        An interactive tool for changing the topics or types of a problem 
        \\\hline
      \texttt{\pytool branch problem.xml -c} &
        Adds a new version to the problem, copied from the previous version.
        Omit the \texttt{-c} to add an empty version. \\\hline
      \texttt{\pytool validate problem.xml} & 
        Finds common mistakes and ensures
        a file conforms to the style guide. All problems you write should
        successfully validate before you put them into the problem folder.
        \\\hline
      \texttt{\pybuild single problem.xml -rms} &
        Renders a single problem (with its metadata, rubric, and solution) into
        \texttt{problem.pdf} \\\hline
      \texttt{\pybuild all --required-topics logic counting -rms} &
        Renders all problems in the problem root that have either logic or
        counting as one of their topics. \\\hline
      \texttt{\pybuild all --authors \`{ }whoami\`{ }} &
        Renders all problems in the problem root that you helped write. 
        \\\hline
      \texttt{\pybuild all --authors \`{ }whoami\`{ } --todo -rms} &
        Renders all problems in the problem root that you wrote, and that still
        need a rubric or a solution. \\\hline
      \texttt{\pybuild all --grep foo bar -rms} &
        Renders all problems in the problem root which contain the text
        `foo' and the text `bar' (case insensitive). \\\hline
      \texttt{\pybuild all --written 2016} &
        Renders all problems in the problem root which were written this year. 
        \\\hline
      \texttt{\pybuild all --written 2016 --verbose} &
        Quick way of seeing what recent problems are not compiling. 
        \\\hline
    \end{tabular}
  \end{center}