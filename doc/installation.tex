\section{Installation}
  To use \LaTeX ML, you must be using some version of Python 2, and must have 
  \texttt{pdflatex} installed on your system. The following Python packages are 
  also required:
  
  \begin{itemize}\itemsep0pt
    \item \texttt{argparse}
    \item \texttt{xml}
  \end{itemize}
  
  The environment variable \texttt{LATEXML\_CONFIG} must be set to the location 
  of a valid configuration XML file in order to run \LaTeX ML. If you are a CS22
  TA (in the \texttt{cs022ta} group) using a department machine, this should be
  done for you -- talk to an HTA if an error appears referring to 
  \texttt{LATEXML\_CONFIG}.
  
  If you are on your home computer, the next few sections will help you set up 
  your configuration file correctly.
  
  \subsection{Requests and Complaints}
    Please tell Nick the moment something doesn't work, or something is harder
    than it should be. This is meant to simplify your life significantly: if
    there is a particular feature that would help you write things more 
    efficiently, Nick would love to at least hear about it (or tell you 
    why it's impossible, that's fun too). If something is confusing or 
    buggy, also say so -- I can't fix something if I don't know it's broken!
    
    Also feel free, if you are good with Python, to submit a pull request 
    in git.
    
  \subsection{Installation on Home Computers}  
    \newcommand\mytilde{\raise.17ex\hbox{$\scriptstyle\sim$}}
    \newcommand\ttquote{\texttt{\char`\"}}
  
    First, open a terminal (Linux or Mac) or Cygwin (Windows - currently do not
    know another option, sorry!). Create a new folder where \LaTeX ML will 
    be installed (\texttt{mkdir \mytilde/latexml}) and switch into it (
    \texttt{cd \mytilde/latexml}). Download \LaTeX ML using git by running 
    \[\texttt{git clone https://www.github.com/nschank/latexml}\]
    
    Open your .bashrc file (\texttt{\mytilde/.bashrc})
    \footnote{In some systems, this is .bash\_profile.} 
    and add the line: 
    \[\texttt{export LATEXML\_CONFIG=\ttquote /home/\textit{login}/latexml/
    config/config.xml\ttquote}\] We also recommend adding aliases to the 
    tools by adding the lines 
    \[\texttt{alias 22build=\ttquote python /home/\textit{login}/latexml/build.py\ttquote}\]
    \[\texttt{alias 22edit=\ttquote python /home/\textit{login}/latexml/edit.py\ttquote}\]
    
    Now open the config.xml file in \texttt{\mytilde/latexml/config}. 
    Change the contents of the \texttt{include} tag to contain the path 
    \[\texttt{/home/\textit{login}/latexml/include/simple22.sty}\] 
    This is the location of 
    the header file inserted into the top of built documents. 
    
    You may optionally
    add a directory of your preference as the \texttt{problemroot} -- this
    is where \texttt{\pybuild all} and \texttt{\pybuild doc} search, but
    it is not necessary for single problems or for editing.
    
    If you will be working with graphics or other resources (e.g. if you
    will be including an image or diagram in a problem), you will need to
    set the \texttt{resourceroot} to an existing directory, and you will
    need to store that image/diagram there. If you do not plan to work
    with resources, the \texttt{resourceroot} does not need to be fixed.
    
    You should now run the tests (\texttt{python tests.py}) to make sure that 
    everything is running smoothly.
    