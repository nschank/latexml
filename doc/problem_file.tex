\section{Writing a Problem}
  To write a problem using \LaTeX ML, you must first understand the basics of the XML encoding being used.
  
  \subsection{Problem XML Specification}
    Here is a sample problem, for the sake of explanation:
    \begin{mdframed}
      \begin{lstlisting}[language=XML,columns=fullflexible,breaklines=true]
<problem>
  <used year="2015">Homework 5</used>
  <used year="2014" private="true">Midterm 1</used>
  <version id="2">
    <author>cjk</author>
    <year>2016</year>
    <topics>number_theory graph_theory</topics>
    <types>proof induction</types>
    <param name="modulus">3</param>
    <dependency>tikz</dependency>
    <body>
      The modulus is \modulus.
    </body>
    <solution>
      TODO
    </solution>
    <rubric>
      TODO
    </rubric>
  </version>
  <version id="1">
    ...
  </version>
</problem>
      \end{lstlisting}
    \end{mdframed}
    
    \textbf{A Problem XML file MUST have root tag \texttt{problem}, which is made up of one or more child \texttt{version}s.} Loosely speaking, two `versions' should be part of the same `problem' if having the solution to one would make writing the other's solution trivial. For example: adding a part to, theming, or changing the numbers in a previous year's problem would call for a new version of the problem. If several different problems are combined into one, that is generally a new problem (rather than a version of any of them).
    
    \textbf{Each version MUST have exactly one year} (in a \texttt{<year>} tag), which may be ``Unknown''. This is the year that the version was written.
    
    \textbf{Each version MUST have at least one author, at least one topic, and at least one type.} Each of these tags accepts a comma and/or whitespace separated list of items; thus,
    \begin{center}
      \texttt{<authors>cjk nschank, kl47</authors>}
    \end{center}
    would be parsed as three authors: cjk, nschank, and kl47. Additional instances of a tag will append to the list, so e.g. multiple \texttt{<author>} tags within a version are allowed.
    
    Each of these fields accepts both the singular and plural of their tagname, purely for convenience. Thus...
    \begin{description}\itemsep0pt
      \item[Authors] \texttt{<author>} or \texttt{<authors>}
      \item[Topic] \texttt{<topic>} or \texttt{<topics>}
      \item[Type] \texttt{<type>} or \texttt{<types>}
    \end{description}
    
    Note that, while either tagame is accepted, the beginning and ending tags must match.
    
    See the next section for an explanation of acceptable values for topic and type.
    
    \textbf{Each version MUST have a body, a solution, and a rubric.} Each of these fields should be filled with arbitrary \LaTeX; whatever you would have put between \texttt{\textbackslash begin\{document\}} and \texttt{\textbackslash end\{document\}} goes here. It is expected that, if a solution or rubric is not complete, the four letters ``TODO'' (case sensitive) should appear somewhere within their text. This allows the \LaTeX ML tools to keep track of any problems that need attention.
    
    As an important note, the characters \&, $<$, and $>$ are special characters within XML. In order for the problem to be parsed, you MUST escape them with the XML sequences \texttt{\&amp;}, \texttt{\&lt;}, and \texttt{\&gt;} respectively. They will be unescaped before being parsed as \LaTeX, so should be treated identically to their corresponding characters (e.g. tables will contain many instances of ``\texttt{\&amp;}'').
    
    \textbf{Each version MUST have an \texttt{id} attribute, unique within the problem, set to a positive integer, such that the newest version has the highest ID.}
    
    A version MAY have zero or more \texttt{param} or, equivalently, \texttt{parameter} tags, which MUST have a name attribute. This is equivalent to temporarily creating a command \texttt{\textbackslash name} which produces the value given in the tag's field. In the provided example, the command ``\texttt{\textbackslash modulus}'' will evaluate to 3. This field is intended for use within problems that can be easily changed without needing to create a new version. As an example, the name of a person or object within a problem should be refactored into a parameter, so that the problem can be changed easily.
    
    A version MAY have zero or more \texttt{dependency} (also allowed: \texttt{dep}, \texttt{deps}, \texttt{dependencies}) tags, each of which should be a comma and/or whitespace-separated list of packages which are required in order to build the problem. \texttt{tikz} is the most commonly included by far. These dependencies will by dynamically imported when building an assignment including this problem.
    
    \texttt{usedin} tags should never be created or edited by hand.
  
  \subsection{Topics}
    A \textbf{topic} is a unit of the class which a question is attempting to focus on. A problem-version may have multiple topics (we encourage it!), and the topics should be kept specific but not overly so. Any unrecognized topic within a topic list is considered a fatal error.
    
    Allowed topics are:
    \begin{itemize}\itemsep0pt
      \item basic\footnote{The `basic' topic should be included on anything that could go on the first homework, before we have really introduced any definitions. As an example, \textit{Prove that $\sqrt 2$ is irrational} would be in the basic topic.}
      \item big\_o
      \item circuits
      \item counting
      \item equivalence\_relations
      \item graph\_theory
      \item logic
      \item mod
      \item number\_theory
      \item pigeonhole
      \item probability
      \item relations
      \item set\_theory
      \item todo\footnote{If the topics field of a version should be looked at later (e.g. you are unsure about one and want to return to it), \textit{always} include todo as a topic.}
    \end{itemize}
    
    If ever confused about what topic(s) a problem covers, consider that the topics field is meant to be organizational. If Carly were making the final exam, and wanted to include a question on this topic, would she want to see this problem? If so, include the topic!
  
  \subsection{Types}
    