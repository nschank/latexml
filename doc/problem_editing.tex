\section{The \textit{Edit} Tool}
  \subsection{Creating a New Problem}
    If you are building a new problem from scratch, you should first make 
    sure that no question like it already exists. You can check this by 
    opening \texttt{/course/cs022/pdf/everything.pdf} and searching 
    for some keywords about your problem. 
    
    Then, you should make a new XML file in your home folder (or wherever
    you wish to work); you should \textit{not} add new files directly into
    the \texttt{/course/cs022/problems/} directory, since your \LaTeX\ 
    errors may cause builds to break. Move your problem into that directory
    (wherever makes the most sense to you and your topic group) once it is
    completed.
    
    There is a simple tool for constructing the skeleton of a new problem 
    XML. On a command line, run: 
    \[\pytool\texttt{new \textit{filename.xml}}\] 
    This command builds a new problem with one version, the current year, 
    your login as the sole 
    author\footnote{If you are on your local machine, and you have a
    different login there, you can add an \texttt{<author>} tag into your
    config file, and that will be the one automatically added instead.}, 
    and other required fields ready to be filled in.
    
    The optional \texttt{-i} flag allows you to select the topics and 
    types of the new file with a convenient interactive command-line interface. 
    
  \subsection{Creating a New Version}
    If you are adding a new version to an existing problem, the edit tool 
    will let you add a new version with minimal effort. Run: 
    \[\pytool\texttt{branch \textit{problem.xml} [-c|-e|-i]}\]
    
    The final flag is optional, and defaults to \texttt{-e}.
    
    \texttt{-e} means the new version will be \underline{e}mpty; the 
    version number will be correctly incremented, the year will be set to 
    the current year, and the sole author will be your login, but all 
    other fields will be empty or ``TODO''.
    
    \texttt{-i} is the same as \texttt{-e}, except it uses the same 
    \underline{i}nteractive interface as \pytool\texttt{new} to let you 
    choose topics and types.
    
    \texttt{-c} \underline{c}opies the previous version's contents 
    (topics, types, parameters, dependencies, body, solution, and rubric),
    and updates the version number, author, and year in the same way that 
    \texttt{-e} does.
    
  \subsection{Editing Topics and Types}
    You can start the interactive topic and type editor on a problem using 
    \[\pytool\texttt{edit \textit{problem.xml}}\]
    
    Optionally, the flag \texttt{--remove-todo} automatically removes the 
    todo topic and type, when fixing such problems.
    
  \subsection{Validating a Problem File}
    To make you feel reassured that your problem XML looks like it's up to 
    the specifications, you can run 
    \[\pytool\texttt{validate \textit{a\_problem.xml}}\] 
    Validation is actually stricter (but more verbose) than the build 
    tools: it looks closely at things like \LaTeX\ style, and likely XML 
    errors that can be very difficult to find (e.g. raw ampersands). 
    You should always run validate, even if your problem builds successfully.
    
    Validate will independently render the body, solution, and rubric of your
    problem, and alert you to the LaTeX errors caused by each rendering.
    