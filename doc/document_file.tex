\section{Assignment Files}
  We don't release problems by themselves -- we release assignments! \LaTeX ML 
  uses a different specification to specify assignments and the problems 
  within them, which simplify the process of building the same set of 
  problems repeatedly.
  
  Assignments are designed in such a way that they may be \textit{finalized}. 
  Finalization is meant to signify that an assignment has been released -- 
  or, more accurately, that an assignment's solutions have been released. 
  When finalized, the assignment updates all of its contained problems with 
  a \texttt{usedin} tag, allowing later years to know how long it has been 
  since a problem was used. It also reifies the version attribute of all 
  problems, so that if a problem is later given a new version, the old 
  homework will not be changed.
  
{
\newcommand\TB{\hspace*{1em}}

  \subsection{Assignment XML Specification}
    \begin{mdframed}
\texttt{
<assignment>\\
\TB<name>}Homework 4\texttt{</name>\\
\TB<year>}2016\texttt{</year>\\
\TB<due>}Tuesday, January 85, 2016\texttt{</due>\\
\TB<problem>}problems/problem1.xml\texttt{</problem>\\
\TB<problem>}problems/problem2.xml\texttt{</problem>\\
\TB<problem version=}``\textit{4}''\texttt{>}problems/problem1.xml\texttt{</problem>\\
</assignment>}
    \end{mdframed}
    
}
    
    \textbf{Note}: The assignment XML specification is still in progress, in 
    the hopes of providing more flexibility in the near future. Thus, any 
    document files not kept in the 22 course folder are not guaranteed to be 
    forward-compatible - Nick will fix any document files kept in the 
    correct location.
    
    All assignments MUST have be given a name (equivalently, a \textit{title})
    in a \texttt{name} (equiv. \texttt{title}) tag. The name is used as the 
    header of the built document, in a very large font.
    
    All assignments MUST have the year that the assignment was made. This 
    year should be accurate--when an assignment is finalized, this is the 
    year the problem will be marked as \texttt{usedin}.
    
    All assignments MUST have a due date, which will be used in the header 
    and the introduction of the document exactly as written. 
    
    All assignments MUST have at least one \texttt{problem}, and an 
    unlimited number (with repeats allowed). These problems will be added in 
    the same order as they are in the XML file, with repetition intact. The 
    \texttt{\textit{version}} attribute can be used to mark which version of 
    the problem is desired. If none is provided, the most recent is 
    automatically used.
    
    The text of a problem tag should be the filepath of an existing, 
    validated problem XML file. (If it is invalid or does exist, the 
    assignment build will fail.) Absolute paths are allowed, and
    relative paths are appended to the problem root.
    
  \subsection{Building}
    You can build an assignment XML with the build tool as follows: 
    \begin{align*}
      \pybuild\texttt{doc }&\texttt{\textit{assign.xml output[.pdf]}}\\
      ~&\texttt{[-s] [-m] [-r] [-k]}
    \end{align*}
    
    The \texttt{-s}, \texttt{-m}, and \texttt{-r} flags control whether 
    solutions, metadata (like filename and topics), and rubrics are printed, 
    respectively. All are optional, and they can be used together independently.
    
    The \texttt{-k} flag prevents the rendered \texttt{.tex} file from
    being deleted.
    
  \subsection{Finalizing}
    Finalizing an assignment is simple, and has no options:
    \[\pytool\texttt{finalize \textit{assign.xml}}\]
    
  \subsection{Private Assignments}
    A \textit{private} assignment is one for which solutions are not publicly
    released -- generally, midterms and finals. You should always mark
    private assignments private, by setting an attribute in the assignment tag: 
    \texttt{<assignment private=}``\textit{true}''\texttt{>}
    
    If a problem was used in a private assignment, it is allowed to appear
    in searches which ask for problems not used in the same year. So, for
    instance, a problem used in the 2014 final may still appear when
    searching with \texttt{--not-used-in 2014}. 