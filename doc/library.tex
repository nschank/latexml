{
\newcommand\comm[1]{\texttt{\textbackslash #1}}

\section{Course Folder Organization}
  \LaTeX ML was created because the 22 course folder was a mess, and was
  built in the hopes of preventing that from happening again. The only way
  to make sure that it \textit{doesn't} happen again is to have considerably
  better institutional memory, even while the staff changes every year.
  
  These homework problems
  stick around for literally decades -- at the time of writing, we have
  problems going back to more than 13 years ago. Codifying standards will
  allow this class to keep improving year by year, instead of constantly 
  struggling to just put a single homework together. With that in
  mind, here are some rules to keep in mind when working in the course
  folder.
  \begin{enumerate}
    \item \textbf{Duplication is your enemy.} The number of person-hours which
      have been wasted by 22 TAs due to problem duplication is truly
      staggering. If you make a new problem, you should be SURE that there is
      not a nearly identical problem lying around.\footnote{Use 
      \texttt{\pybuild --grep}!} And if you are modernizing an old problem,
      you should NEVER copy and paste the file -- make a new version
      instead.
    \item \textbf{Organize conceptually.} In particular, you should not
      organize the entire problem root by year: organizing by year will
      make duplication far too likely, and will bias the staff towards
      using more recent problems.
      
      Instead, we recommend the following. Break up the problems into a
      few directories based on topic (e.g. graph theory, number theory...).
      Each of these folders should have a subdirectory called \texttt{old},
      which contains all problems that have not been looked at by the 
      \textit{current year}'s staff. When a problem is looked at and fixed
      up, it should be moved into the main topic directory. Conversely,
      at the end of the year, all problems should be put back into the
      \texttt{old} subdirectory of their topic.
    \item \textbf{Finalize assignments.} If you finalize your assignments,
      later years will be able to much more easily find the problems they
      can use. Finalizing an assignment marks each problem as used in
      the correct year, and the build tool can filter out problems used
      too recently, e.g. \texttt{--not-used-in 2014 2015}.
      
      Take advantage of \textit{private} assignments, which allow you to mark
      an assignment as a midterm/exam which did not have solutions released.
      The problems within it can be reused soon afterwards, and the finalize
      tool will mark the problems as used privately (allowing the build tool
      to not filted it out).
    \item \textbf{Organize resources.} The resource directory, found at
      \texttt{/course/cs022/resources}, should be kept neat and tidy.
      Contrary to the problems folder, the resources folder \textit{should
      be organized by year}. In fact, more specifically, you should never
      edit or alter ANYTHING in a previous year's resources. Please instead
      copy old resources into your own year.
      
      Reasoning: images and graphics have historically been very difficult to
      keep straight, because everything keeps moving around. Old problems
      would reference some old image, but it had been renamed since, and 
      somebody edited the image too, so now the problem makes no sense. No
      more! In later years, old versions will reference an image which
      has the same filename and same contents as when the problem was
      written.
    \item \textbf{Be lenient when adding topics.} It is much better,
      organizationally, if you include all topics that are relevant to your
      problem... even if it is not the focus of that problem. By combined
      usage of the \texttt{--allowed-topics} and \texttt{--required-topics}
      tags, you will have a much easier time finding appropriate problems.
      
      As an example, let's say you write a question that focuses on graph
      theory, but incorporates some set theory, and probability: you should
      DEFINITELY include all three of them as topics. If we are looking for a
      question on the set theory homework, it is easy to not include graph
      theory questions, so the inclusion of the topic does no harm. But the
      extra information is still useful if we are looking for a question that
      connects back to set theory when making a graph theory homework.
    \item \textbf{Validate constantly.} Spam the validation tool. It is the
      most important and rapid way of making sure that your problem is up to
      consistently set 
      standards, which will save time for everyone overall. (Not to mention it
      is fairly good at finding likely errors.)
  \end{enumerate}

\section{\LaTeX\ Style}
  In order to keep the formatting of \LaTeX\ consistent between problems, we 
  have a few special macros and stylistic choices you are \textit{required}
  to use.
  
  \begin{enumerate}
    \item You should \textbf{ALWAYS} use the environments \texttt{22itemize} 
      and \texttt{22enumerate} rather than \texttt{itemize} and 
      \texttt{enumerate} respectively. Specifically, these environments 
      change the spacing between items, and use a consistent lettering scheme.
    \item When providing a hint or a note, you \textbf{MUST} use the 
      commands \texttt{hint} and \texttt{note}, as in 
      \[\texttt{\textbackslash hint\{Think about the pigeonhole principle.\}}\] 
      This allows us to change how all hints and notes look without seeking 
      them out individually.
    \item The newline commands (e.g. \texttt{\textbackslash\textbackslash}) 
      should \textbf{NEVER} be used to make paragraphs or for spacing of text. 
      To make a new paragraph, leave a blank line between two blocks of text, 
      as in: 
      \begin{verbatim}
\begin{document}
  First paragraph.
  
  Second paragraph.
\end{document}
      \end{verbatim}
      
      This does \textit{not} apply to tables or other such environments; you
      should use \texttt{\textbackslash\textbackslash} as you normally would.
    \item All parts to a question should end with a period 
      (or question/exclamation mark). You'd be surprised how often people 
      forget this!
    \item \textbf{NEVER} use the double-dollar-sign notation 
      (e.g. \texttt{\$\$x\$\$}); in fact, don't do it in real life either. 
      It is error-prone and difficult to identify using i.e. regexes. Use 
      \texttt{\textbackslash[x\textbackslash]} instead.
    
      (Single dollar signs are still the preferred notation for inline math.)
    \item \textbf{ALWAYS} use \comm{pmod} to write things in mod notation. 
      
      \textbf{Example}: \texttt{3\comm{equiv} 4\comm{pmod} 5} compiles to 
        $3\equiv 4\pmod 5$.
      
      \textbf{Wrong}: \texttt{3\comm{equiv} 4\comm{mod} 5} compiles to 
        $3\equiv 4\mod 5$, has weird spacing and we wish to maintain 
        consistency.
      
      \textbf{Very Wrong}: \texttt{3=4 mod 5} compiles to $3=4 mod 5$ and 
        will lead to spontaneous crying among the staff.   
    \item Keep lines under 80 characters to remain editor friendly.
      The validator enforces this too!
    \item Use \comm{setminus} to write the `backslash' used in set algebra;
      the text backslash character does not provide the right spacing, and is 
      less readable in the \LaTeX.
  \end{enumerate}
  
  \subsection{22-Specific Macros}
    Here is a small table of other convenience macros that are available to you.
    We \textbf{highly recommend} them, as they cut down on (1) useless 
    clutter, and (2) minor typos.
    
    \begin{center}
      \begin{tabular}{|c|c|}
        \hline
        \textbf{Symbol} & \textbf{Command} \\\hline
        $\mathbb{N}$ & \texttt{\textbackslash N} \\\hline
        $\mathbb{Z}$ & \texttt{\textbackslash Z} \\\hline
        $\mathbb{R}$ & \texttt{\textbackslash R} \\\hline
        $\mathbb{Q}$ & \texttt{\textbackslash Q} \\\hline
        $\operatorname{E}[X]$ & \texttt{\textbackslash E[X]} \\\hline
        $\operatorname{V}(X)$ & \texttt{\textbackslash V(X)} \\\hline
        $\{x\;|\;x\in X\}$ & \texttt{\textbackslash setbuilder\{x\}\{x\textbackslash in\ X\}} \\\hline
        $\mathcal{P}$ & \texttt{\textbackslash Pow} \\\hline
        $\Pr[\ ]$ & \texttt{\textbackslash Pr[\ ]} \\\hline
      \end{tabular}
    \end{center}
    
    Note that the \texttt{setbuilder} command does not deal with multiple 
    line equations.
  
  \subsection{\LaTeX\ Tips}
    Here are some tips to writing \LaTeX\ more effectively. 
    \begin{itemize}
      \item The commands \comm{left} and \comm{right} can be used with delimiters like \texttt{\{\}}, \texttt{()}, and \texttt{[]} to make them grow with their contents. So, for instance, \texttt{[\comm{binom}\{n\}\{r\}]} looks like \[[\binom nr]\] while \texttt{\comm{left}[\comm{binom}\{n\}\{r\}\comm{right}]} looks like \[\left[\binom nr\right]\]
      
      The matching command \comm{middle} can be used with the pipe character $|$ to produce effective set builder notation (and that's exactly how \comm{setbuilder} does it!)
      \item Trying to write implies statements or iff statements? You are looking for \comm{Rightarrow} ($\Rightarrow$) and \comm{Leftrightarrow} ($\Leftrightarrow$).
      \item Be careful with XML special characters \&, $<$, and $>$! See the section on special characters to avoid confusing XML parse errors. (Thankfully, XML color coding should make these errors relatively easy to spot.)
    \end{itemize}
    
  \subsection{Adding to This List}
    Contact Nick and offer more tips or stylistic decisions, if you think 
    they are important! We are very willing to add useful commands into our 
    built-in macros.
  
  
}
  
  
  