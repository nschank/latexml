{
\newcommand\comm[1]{\texttt{\textbackslash #1}}

\section{\LaTeX\ Style}
  In order to keep the formatting of \LaTeX\ consistent between problems, we have a few special macros and stylistic choices you are \textit{required} to use.
  
  \begin{enumerate}
    \item You should \textbf{ALWAYS} use the environments \texttt{22itemize} and \texttt{22enumerate} rather than \texttt{itemize} and \texttt{enumerate} respectively. Specifically, these environments change the spacing between items, and use a consistent lettering scheme.
    \item When providing a hint or a note, you \textbf{MUST} use the commands \texttt{hint} and \texttt{note}, as in \[\texttt{\textbackslash hint\{Think about the pigeonhole principle.\}}\] This allows us to change how all hints and notes look without seeking them out individually.
    \item The newline commands (e.g. \texttt{\textbackslash\textbackslash}) should \textbf{NEVER} be used to make paragraphs or for spacing. To make a new paragraph, leave a blank line between two blocks of text, as in: \begin{verbatim}
\begin{document}
  First paragraph.
  
  Second paragraph.
\end{document}
    \end{verbatim}
    \item All parts to a question should end with a period (or question/exclamation mark). You'd be surprised how often people forget this!
    \item \textbf{NEVER} use the double-dollar-sign notation (e.g. \texttt{\$\$x\$\$}); in fact, don't do it in real life either. It is error-prone and difficult to identify using i.e. regexes. Use \texttt{\textbackslash[x\textbackslash]} instead.
    
    (Single dollar signs are still the preferred notation for inline math.)
    \item \textbf{ALWAYS} use \comm{pmod} to write things in mod notation. 
      
      \textbf{Example}: \texttt{3\comm{equiv} 4\comm{pmod} 5} compiles to $3\equiv 4\pmod 5$.
      
      \textbf{Wrong}: \texttt{3\comm{equiv} 4\comm{mod} 5} compiles to $3\equiv 4\mod 5$, has weird spacing and we wish to maintain consistency.
      
      \textbf{Very Wrong}: \texttt{3=4 mod 5} compiles to $3=4 mod 5$ and will lead to spontaneous crying among the staff.   
    \item Keep lines under 80 characters to remain editor friendly.
      The validator will enforce this soon!
  \end{enumerate}
  
  \subsection{22-Specific Macros}
    Here is a small table of other convenience macros that are available to you. We \textbf{highly recommend} them, as they cut down on (1) useless clutter, and (2) minor typos.
    
    \begin{center}
      \begin{tabular}{|c|c|}
        \hline
        \textbf{Symbol} & \textbf{Command} \\\hline
        $\mathbb{N}$ & \texttt{\textbackslash N} \\\hline
        $\mathbb{Z}$ & \texttt{\textbackslash Z} \\\hline
        $\mathbb{R}$ & \texttt{\textbackslash R} \\\hline
        $\mathbb{Q}$ & \texttt{\textbackslash Q} \\\hline
        $\operatorname{E}[X]$ & \texttt{\textbackslash E[X]} \\\hline
        $\operatorname{V}(X)$ & \texttt{\textbackslash V(X)} \\\hline
        $\{x\;|\;x\in X\}$ & \texttt{\textbackslash setbuilder\{x\}\{x\textbackslash in\ X\}} \\\hline
        $\mathcal{P}$ & \texttt{\textbackslash Pow} \\\hline
        $\Pr[\ ]$ & \texttt{\textbackslash Pr[\ ]} \\\hline
      \end{tabular}
    \end{center}
    
    Note that the \texttt{setbuilder} command does not deal with multiple line equations.
  
  \subsection{\LaTeX\ Tips}
    Here are some tips to writing \LaTeX\ more effectively. 
    \begin{itemize}
      \item The commands \comm{left} and \comm{right} can be used with delimiters like \texttt{\{\}}, \texttt{()}, and \texttt{[]} to make them grow with their contents. So, for instance, \texttt{[\comm{binom}\{n\}\{r\}]} looks like \[[\binom nr]\] while \texttt{\comm{left}[\comm{binom}\{n\}\{r\}\comm{right}]} looks like \[\left[\binom nr\right]\]
      
      The matching command \comm{middle} can be used with the pipe character $|$ to produce effective set builder notation (and that's exactly how \comm{setbuilder} does it!)
      \item Trying to write implies statements or iff statements? You are looking for \comm{Rightarrow} ($\Rightarrow$) and \comm{Leftrightarrow} ($\Leftrightarrow$).
      \item Be careful with XML special characters \&, $<$, and $>$! See the section on special characters to avoid confusing XML parse errors. (Thankfully, XML color coding should make these errors relatively easy to spot.)
    \end{itemize}
    
  \subsection{Adding to This List}
    Contact Nick and offer more tips or stylistic decisions, if you think they are important! We are very willing to add useful commands into our built-in macros.
  
  
}
  
  
  