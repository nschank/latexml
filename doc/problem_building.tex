\section{The \textit{Build} Tool}
  \subsection{Building Specific Problems}
    You can build any problem(s) into a PDF in one step by running the following command:
    \begin{align*}\pybuild\texttt{problems }&\texttt{\textit{output[.pdf]} problem1.xml [problem2.xml \dots]}\\
        ~&\texttt{[-s] [-m] [-r] [--title ``TITLE'']}\end{align*}
    
    The build script is overwrite-safe and tolerant of multiple people building the same pdf simultaneously. In particular, it will build a randomly named temporary file and check for overwriting afterwards, allowing you to specify another name or to overwrite if necessary. The generated \texttt{.tex} file is removed, as are the logs and auxiliary files -- if you need these for some reason, let Nick know and he'll add an option to keep them. Exception: the \texttt{.tex} file is not removed if there were \LaTeX\ errors, so that you can find where the error occurred.
    
    The most recent version (by highest ID) of each problem will be built in order, with repetition if requested. The header on top is temporary and will be removed in a later version of the builder. To change the title of the output, use the \texttt{--title} flag (which takes one argument, so may need quotes).
    
    The \texttt{-s}, \texttt{-m}, and \texttt{-r} flags control whether solutions, metadata (like filename and topics), and rubrics are printed, respectively. All are optional, and they can be used together arbitrarily.
    
    An invalid problem XML will be skipped and a warning will be printed, but the build will continue.
    
  \subsection{Building All Problems of a Type}
    Given a directory full of problem XML files, the builder script can search all of them individually and build those that meet a set of criteria. You should run
    \begin{align*}\pybuild\texttt{from }&\texttt{\textit{problems/\ \ output[.pdf]} [CRITERIA]}\\~&\texttt{[-s] [-m] [-r] [--title ``TITLE'']}\end{align*}
    If no criteria are given, the above command builds every problem in the directory \texttt{problems/}, with the four optional flags acting just as in the previous section.
    
    The criteria for inclusion can be specified with any or all of the following flags, where a problem will be included only if it satisfies all specified flags.
    \begin{description}
      \item[\texttt{--allowed-topics}] After this flag, include one or more topic names, separated by spaces. A problem will be included only if its topics are a subset of this list. \textbf{Common use case}: List all topics that the students have learned so far, and you will never include problems that they are not qualified to answer.
      \item[\texttt{--grep}] After this flag, include one or more search terms separated by whitespace. A problem will be included only if each search term is present in either the body, solution, or rubric. The search is case insensitive, but note that it is searching the raw \LaTeX\ so is sensitive to formatting and whitespace.
      \item[\texttt{--required-topics}] After this flag, include one or more topic names, separated by spaces. A problem will be included only if it has at least one required topic. Note that it does not need to possess \textit{all} required topics. \textbf{Common use case}: List the topics that the homework should focus on to get a good set of tentative problems for a homework.
      \item[\texttt{--required-types}] After this flag, include one or more topic names, separated by spaces. A problem will be included only if it has at least one required type. Note that it does not need to possess \textit{all} required types. \textbf{Common use cases}: Finding (for instance) an inductive proof on a particular topic; listing all `pieces' related to a topic, to get inspiration; finding things that need work.
      \item[\texttt{--todo}] If present, only includes problems that have ``TODO'' in their solution or rubric. Meant to make it easier to find problems still being worked on.
    \end{description}