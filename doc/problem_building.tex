\section{The \textit{Build} Tool}
  \subsection{Building Specific Problems}
    You can build any problem(s) into a PDF in one step by running the 
    following command:
    \begin{align*}\pybuild\texttt{problems }&\texttt{\textit{output[.pdf]} problem1.xml [problem2.xml \dots]}\\
        ~&\texttt{[-s] [-m] [-r] [-k] [--title ``TITLE''] [--verbose]}\end{align*}
        
    If making only a single problem, you can use the shortcut:
    \[\pybuild\texttt{single \textit{name.xml} 
        [-s] [-m] [-r] [-k] [--title ``TITLE''] [--verbose]}\]
    which simply names the output \texttt{name.pdf} (i.e. the same name as 
    the problem file, except with .pdf instead of .xml).
    
    The build script is overwrite-safe and tolerant of multiple people 
    building the same pdf simultaneously. In particular, it will build a 
    randomly named temporary file and check for overwriting afterwards, 
    allowing you to specify another name or to overwrite if necessary. The 
    generated \texttt{.tex} file is removed, as are the logs and auxiliary 
    files -- if you need these for some reason, let Nick know and he'll 
    add an option to keep them. Exception: the \texttt{.tex} file is not 
    removed if there were \LaTeX\ errors, so that you can find where the 
    error occurred.\footnote{As a quick tip: use the \texttt{-m} flag to 
    more easily find what files contain \LaTeX\ errors.}
    
    The most recent version (by highest ID) of each problem will be built 
    in order, with repetition if requested. To change the 
    title of the output, use the \texttt{--title} flag (which takes one 
    argument, so may need quotes).
    
    The \texttt{-s}, \texttt{-m}, and \texttt{-r} flags control whether 
    solutions, metadata (like filename and topics), and rubrics are 
    printed, respectively. All are optional, and they can be used together 
    independently.
    
    The \texttt{-k} flag (for \textit{keep}) tells the tool not to delete the
    intermediary \texttt{.tex} file.
    
    An invalid problem XML will be skipped silently, but the build will
    continue. Use the \texttt{--verbose} flag to have the tool print out 
    what files are being added or skipped and why.
  \subsection{Building All Problems of a Type}
    The builder script can search all of the problems within the root problem 
    directory\footnote{This is set for you in the department machines, within 
    \texttt{/course/cs022/problems}; on your local machine, you can set it 
    within the config file by changing the path set as the 
    \texttt{problemroot}.} 
    individually, building those that meet a set of criteria. You should run
    \begin{align*}
      \pybuild\texttt{all }&\texttt{\textit{output[.pdf]} [CRITERIA]}\\
      ~&\texttt{[-s] [-m] [-r] [-k] [--title ``TITLE''] [--verbose]}
    \end{align*}
    If no criteria are given, the above command builds every problem in 
    the problem root directory, with the four optional flags acting just 
    as in the previous section.
    
    Alternatively, you can specify a different directory to search using 
    the \texttt{from} option:
    \begin{align*}
      \pybuild\texttt{from }&\texttt{\textit{problems/\ \ output[.pdf]} [CRITERIA]}\\
      ~&\texttt{[-s] [-m] [-r] [-k] [--title ``TITLE''] [--verbose]}
    \end{align*}
    This would search everything within the \texttt{problems/} directory.
    
    The criteria for inclusion can be specified with any or all of the 
    following flags, where a problem will be included only if it satisfies 
    all specified flags.
    \begin{description}
      \item[\texttt{--allowed-topics}] 
        After this flag, include one or more topic names, separated by 
        spaces. A problem will be included only if its topics are a subset 
        of this list. 
      
        \textbf{Common use case}: List all topics that the students have 
        learned so far, and you will never include problems that they are 
        not qualified to answer.
      \item[\texttt{--authors}] 
        After this flag, include one or more 
        logins separated by whitespace. A problem will be included only if
        it shares at least one author with this list.
      \item[\texttt{--grep}] 
        After this flag, include one or more search 
        terms separated by whitespace. A problem will be included only if 
        each search term is present in either the body, solution, or rubric. 
        The search is case-insensitive, but note that it is searching the 
        raw \LaTeX, so it \textit{is} sensitive to formatting and whitespace.
      \item[\texttt{--not-used-in}] 
        If present, only includes problems that 
        were not used in any of the years given. Use `none' to only 
        include problems that have been used in at least one homework.
      \item[\texttt{--required-topics}] 
        After this flag, include one or more 
        topic names, separated by spaces. A problem will be included only 
        if it has at least one required topic. Note that it does not need 
        to possess \textit{all} required topics. 
      
        \textbf{Common use case}: List the topics that the homework should 
        focus on to get a good set of tentative problems for a homework.
      \item[\texttt{--required-types}] 
        After this flag, include one or more type
        names, separated by spaces. A problem will be included only if it has 
        at least one required type. Note that it does not need to possess 
        \textit{all} required types. 
        
        \textbf{Common use cases}: Finding (for instance) an inductive proof 
        on a particular topic; listing all `pieces' related to a topic, to get
        inspiration; finding things that need work.
      \item[\texttt{--todo}] 
        Standalone flag (i.e. does not take any arguments). If present, only 
        includes problems that have ``TODO'' in their 
        solution or rubric. Meant to make it easier to find problems still 
        being worked on.
      \item[\texttt{--used-in}] 
        After this flag, include one or more years separated by whitespace.
        If present, only includes problems that were used in any of the years 
        given. Use `none' to include problems that have never been used. 
      \item[\texttt{--written}] 
        After this flag, include one or more years separated by whitespace. 
        If present, only includes problems that were written in any of the 
        years given.
    \end{description}
    
  The command \[\texttt{\pytool list [CRITERIA] [--verbose]}\] is a quick way to
  see the names of all files that would be included in a build.